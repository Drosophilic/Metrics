\documentclass[english]{article}
\usepackage[utf8]{inputenc}
\usepackage[T1]{fontenc}
\usepackage{babel}
\usepackage{color}
%\usepackage{amsmath}
\usepackage{graphicx}
\usepackage{fancyhdr}
\usepackage{longtable}
%\usepackage{makecell}  % $ sudo tlmgr install makecell
%\pagestyle{fancy}
%\fancyhf{}
\renewcommand{\headrulewidth}{0pt}
\setlength{\headheight}{20pt}

\begin{document}

% \title{FAIR Metric FM-A1.1}
\title{ PYTHON getTitle}

% \author{Mark D. Wilkinson, Susanna-Assunta Sansone, \\Erik Schultes, Peter Doorn,\\ Luiz Olavo Bonino da Silva Santos, Michel Dumontier}
\author{ PYTHON getAuthors }

\maketitle

\newpage

\begin{longtable}{|p{5cm}|p{9cm}|}

\hline
\emph{FIELD} & \emph{DESCRIPTION} \\

\hline
Metric Identifier &
% FM-A1.1: \verb"https://purl.org/fair-metrics/FM_A1.1"
[PYTHON getShortID]: \verb"[PYTHON getID]"
\\

\hline
Metric Name &
Access Protocol
PYTHON getShortTitle

\\

\hline
To which principle does it apply? &

% A1.1 - the protocol is open, free, and universally implementable
PYTHON getTopicTitle + getTopicDescription

\\

\hline
What is being measured? &

% The nature and use limitations of the access protocol.
PYTHON getMeasuring


\\

\hline
Why should we measure it? &

% Access to a resource may be limited by the specified communication protocol. In particular, we are worried about access to technical specifications and any costs associated with implementing the protocol. Protocols that are closed source or that have royalties associated with them could prevent users from being able to obtain the resource.
PYTHON getReason

\\

\hline
What must be provided? &

% i) A URL to the description of the protocol\newline
% ii) true/false as to whether the protocol is open source\newline
% iii) true/false as to whether the protocol is (royalty) free\newline
PYTHON getRequirements

\\

\hline
How do we measure it? &

% Do an HTTP get on the URL to see if it returns a valid document. Ideally, we would have a universal database of communication protocols from which we can check this URL. We also check whether questions 2 and 3 are true or false.
PYTHON getProcedure

\\

\hline
What is a valid result? &

% The HTTP GET on the URL should return a 200,202,203 or 206 HTTP response after resolving all and any prior redirects. e.g. 301 -> 302 -> 200 OK. The other two should be true/false.
PYTHON getValidation

\\

\hline
For which digital resource(s) is this relevant? &
% All
PYTHON getRelevance

\\

\hline
Examples of their application across types of digital resource &
% None
PYTHON getExamples

\\

\hline

Comments &
% None
PYTHON getComments

\\
\hline

\end{longtable}
\end{document}
