\documentclass[english]{article}
\usepackage[utf8]{inputenc}
\usepackage{babel}
\usepackage{color}
%\usepackage{amsmath}
\usepackage{graphicx}
\usepackage{fancyhdr}
\usepackage{longtable}
%\usepackage{makecell}  % $ sudo tlmgr install makecell
%\pagestyle{fancy}
%\fancyhf{}
\renewcommand{\headrulewidth}{0pt}
\setlength{\headheight}{20pt} 

\begin{document}

\title{FAIR Metric FM-A2}

\author{Mark D. Wilkinson, Susanna-Assunta Sansone, \\Erik Schultes, Peter Doorn,\\ 
Luiz Olavo Bonino da Silva Santos, Michel Dumontier}

\maketitle

\newpage





\begin{longtable}{|p{5cm}|p{9cm}|}


\hline
\emph{FIELD} & \emph{DESCRIPTION} \\
\hline
Metric Identifier &   FM-A2: \verb"https://purl.org/fair-metrics/FM_A2"
\\


\hline
Metric Name &   


Metadata Longevity


 \\



\hline
To which principle does it apply? &   


A2 - metadata are accessible, even when the data are no longer available

\\



\hline
What is being measured? & 


The existence of metadata even in the absence/removal of data


\\



\hline
Why should we measure it? & 



Cross-references to data from third-party’s FAIR data and metadata will naturally degrade over time, and become “stale links”.  In such cases, it is important for FAIR providers to continue to provide descriptors of what the data was to assist in the continued interpretation of those third-party data.  As per FAIR Principle F3, this metadata remains discoverable, even in the absence of the data, because it contains an explicit reference to the IRI of the data.
  
\\



\hline
What must be provided? &  


URL to a formal metadata longevity plan


\\



\hline
How do we measure it? &  

Resolve the URL


\\



\hline
What is a valid result? &  



- Successful resolution\newline 
- Returns a document that represents a plan or policy of some kind\newline 
- Preferably certified (e.g. DSA)\newline 



\\



\hline
For which digital resource(s) is this relevant? &  All metadata\\



\hline
Examples of their application across types of digital resource &  None

\\



\hline

Comments & None 

\\
\hline

\end{longtable}



\end{document}