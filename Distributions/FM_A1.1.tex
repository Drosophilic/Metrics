\documentclass[english]{article}
\usepackage[utf8]{inputenc}
\usepackage[T1]{fontenc}
\usepackage{babel}
\usepackage{color}
%\usepackage{amsmath}
\usepackage{graphicx}
\usepackage{fancyhdr}
\usepackage{longtable}
%\usepackage{makecell}  % $ sudo tlmgr install makecell
%\pagestyle{fancy}
%\fancyhf{}
\renewcommand{\headrulewidth}{0pt}
\setlength{\headheight}{20pt} 

\begin{document}

\title{FAIR Metric FM-A1.1}

\author{Mark D. Wilkinson, Susanna-Assunta Sansone, \\Erik Schultes, Peter Doorn,\\ 
Luiz Olavo Bonino da Silva Santos, Michel Dumontier}

\maketitle

\newpage





\begin{longtable}{|p{5cm}|p{9cm}|}


\hline
\emph{FIELD} & \emph{DESCRIPTION} \\
\hline
Metric Identifier &   FM-A1.1: \verb"https://purl.org/fair-metrics/FM_A1.1"
 \\


\hline
Metric Name &   



Access Protocol




 \\



\hline
To which principle does it apply? &   

A1.1 - the protocol is open, free, and universally implementable

\\



\hline
What is being measured? & 


The nature and use limitations of the access protocol.


\\



\hline
Why should we measure it? & 



Access to a resource may be limited by the specified communication protocol. In particular, we are worried about access to technical specifications and any costs associated with implementing the protocol. Protocols that are closed source or that have royalties associated with them could prevent users from being able to obtain the resource.
  
\\



\hline
What must be provided? &  


i) A URL to the description of the protocol\newline  
ii) true/false as to whether the protocol is open source\newline 
iii) true/false as to whether the protocol is (royalty) free\newline 




\\



\hline
How do we measure it? &  

Do an HTTP get on the URL to see if it returns a valid document. Ideally, we would have a universal database of communication protocols from which we can check this URL. We also check whether questions 2 and 3 are true or false.  


\\



\hline
What is a valid result? &  



The HTTP GET on the URL should return HTTP 200. The other two should be true/false.


\\



\hline
For which digital resource(s) is this relevant? &  All\\



\hline
Examples of their application across types of digital resource &  None

\\



\hline

Comments & None 

\\
\hline

\end{longtable}



\end{document}