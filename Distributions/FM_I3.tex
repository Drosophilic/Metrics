\documentclass[english]{article}
\usepackage[utf8]{inputenc}
\usepackage[T1]{fontenc}
\usepackage{babel}
\usepackage{color}
%\usepackage{amsmath}
\usepackage{graphicx}
\usepackage{fancyhdr}
\usepackage{longtable}
%\usepackage{makecell}  % $ sudo tlmgr install makecell
%\pagestyle{fancy}
%\fancyhf{}
\renewcommand{\headrulewidth}{0pt}
\setlength{\headheight}{20pt} 

\begin{document}

\title{FAIR Metric FM-I3}

\author{Mark D. Wilkinson, Susanna-Assunta Sansone, \\Erik Schultes, Peter Doorn,\\ 
Luiz Olavo Bonino da Silva Santos, Michel Dumontier}

\maketitle

\newpage





\begin{longtable}{|p{5cm}|p{9cm}|}


\hline
\emph{FIELD} & \emph{DESCRIPTION} \\
\hline
Metric Identifier &   FM-I3
\\


\hline
Metric Name &   



Use Qualified References


 \\



\hline
To which principle does it apply? &   


I3 - (meta)data include qualified references to other (meta)data

\\



\hline
What is being measured? & 


Relationships within (meta)data, and between local and third-party data, have explicit and ‘useful’ semantic meaning


\\



\hline
Why should we measure it? & 


One of the reasons that HTML is not suitable for machine-readable knowledge representation is that the hyperlinks between one document and another do not explain the nature of the relationship - it is “unqualified”.  For Interoperability, the relationships within and between data must be more semantically rich than “is (somehow) related to”.\newline 
\newline 
Numerous ontologies include richer relationships that can be used for this purpose, at various levels of domain-specificity.  For example, the use of skos for terminologies (e.g. exact matches), or the use of SIO for genomics (e.g. “has phenotype” for the relationship between a variant and its phenotypic consequences).\newline 
\newline 
Similarly, dbxrefs must be predicated with a meaningful relationship  what is the nature of the cross-reference?\newline 
\newline 
Finally, data silos thwart interoperability.  Thus, we should reasonably expect that some of the references/relations point outwards to other resources, owned by third-parties; this is one of the requirements for 5 star linked data. \newline 

  
\\



\hline
What must be provided? &  


Linksets (in the formal sense) representing part or all of your resource



\\



\hline
How do we measure it? &  


The linksets must have qualified references

At least one of the links must be in a different Web domain (or the equivalent of a different namespace for non-URI identifiers)



\\



\hline
What is a valid result? &  


- References are qualified\newline
- Qualities are beyond “Xref” or “is related to”\newline
- One of the cross-references points outwards to a distinct namespace\newline


\\



\hline
For which digital resource(s) is this relevant? &  All\\



\hline
Examples of their application across types of digital resource &  None

\\



\hline

Comments & 


\\
\hline

\end{longtable}



\end{document}