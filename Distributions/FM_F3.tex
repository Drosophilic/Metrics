\documentclass[english]{article}
\usepackage[utf8]{inputenc}
\usepackage[T1]{fontenc}
\usepackage{babel}
\usepackage{color}
%\usepackage{amsmath}
\usepackage{graphicx}
\usepackage{fancyhdr}
\usepackage{longtable}
%\usepackage{makecell}  % $ sudo tlmgr install makecell
%\pagestyle{fancy}
%\fancyhf{}
\renewcommand{\headrulewidth}{0pt}
\setlength{\headheight}{20pt} 

\begin{document}

\title{FAIR Metric FM-F3}

\author{Mark D. Wilkinson, Susanna-Assunta Sansone, \\Erik Schultes, Peter Doorn,\\ 
Luiz Olavo Bonino da Silva Santos, Michel Dumontier}

\maketitle

\newpage





\begin{longtable}{|p{5cm}|p{9cm}|}


\hline
\emph{FIELD} & \emph{DESCRIPTION} \\
\hline
Metric Identifier &   FM-F3: \verb"https://purl.org/fair-metrics/FM_F3"
 \\


\hline
Metric Name &   

Resource Identifier in Metadata


 \\



\hline
To which principle does it apply? &   
F3 - metadata clearly and explicitly include the identifier of the data it describes
\\



\hline
What is being measured? & 


Whether the metadata document contains the globally unique and persistent identifier for the digital resource.

\\



\hline
Why should we measure it? & 



The discovery of digital object should be possible from its metadata. For this to happen, the metadata must explicitly contain the identifier for the digital resource it describes. 
A metadata document should also not result in ambiguity about the digital object it is describing. This can be assured if the metadata document explicitly refers to the digital object by its IRI.

  
\\



\hline
What must be provided? &  


The URL of the metadata and the IRI of the digital resource it describes.


 \\



\hline
How do we measure it? &  

Parsing the metadata for the given digital resource IRI.

\\



\hline
What is a valid result? &  


Present or absent


\\



\hline
For which digital resource(s) is this relevant? &  All\\



\hline
Examples of their application across types of digital resource &  

None


\\



\hline

Comments & 


None

 \\ 
\hline

\end{longtable}



\end{document}